\chapter{Getting OpenSSL}

There are multiple ways to get hold of OpenSSL. This chapter will cover
how to compile and install OpenSSL from source, as well as discuss various
options for obtaining OpenSSL in pre-built binary form.

\section{Understanding OpenSSL version numbers}

The OpenSSL project (\url{https://www.openssl.org}) develops and
maintains the OpenSSL source code. The project stores all of the source
code in a source code control system known as git. Periodically the
project will generate a new \emph{release}. This is a snapshot of the source
code at a given point in time that is collected together into a single archive
file and made available for download.

At any one time there may be multiple versions of OpenSSL that are receiving
support by the project team. Support in this context means that bugs are
fixed and security issues are investigated, reported and fixed as appropriate.

For example OpenSSL 1.1.0 and OpenSSL 1.0.2 are two versions of OpenSSL. Each
version of OpenSSL may have multiple releases associated with it. These are
identified by a letter after the version number. The very first release for
a version has no letter. For example the first release for OpenSSL 1.1.0 has
the version number 1.1.0, the second one is 1.1.0a, the third one is 1.1.0b,
and so on. The letter releases contain bug and security fixes for the given
OpenSSL version (but never new features). To ensure that you always have the
latest fixes you should always try to use the latest letter release that is
available for the OpenSSL version that you are using. Letter releases for a
version are always fully backward compatible.

Versions of OpenSSL which have the same first two numbers are backwards source
and binary compatible with each other\footnote{This only applies if both versions
have been compiled with the same options}. For example 1.0.2 is backwards
source and binary compatible with 1.0.1 and 1.0.0. This means upgrading from
one of these versions to a later one should simply be a matter of upgrading the
library itself. No recompilation of application binaries using OpenSSL should be
necessary. If recompilation of applications is done then they should continue to
work as before. New features will typically have been added if the third number
has been increased.

Versions of OpenSSL which change the middle number are not backwards source or
binary compatible with each other. For example 1.1.0 is not fully backwards
source or binary compatible with 1.0.2. The changes required to upgrade an
application that works with 1.0.2 for 1.1.0 are relatively small and straight
forward though.

The OpenSSL project has not defined what it means if the first number changes,
but we should assume that this would signal some very significant change. The
project has no current plans for a version that changes this number.

OpenSSL versions are supported for a given period of time. Some OpenSSL versions
are designated as Long Term Support (LTS) versions. Non-LTS versions are supported
for 2 years. LTS versions are supported for 5 years. During the final year of
any support period the version only receives security fixes (but not bug fixes).
OpenSSL 1.0.2 is an LTS release.

Note that it is usually possible to have multiple versions of OpenSSL installed on
the same system at the same time.

New applications that are being developed to use OpenSSL should typically use the
latest version of OpenSSL available. This guide describes the use of OpenSSL 1.1.x
(where "x" is any number). It does not describe how to use OpenSSL 1.0.2 (or earlier)
which has a number of differences.

In addition to the main OpenSSL versions the OpenSSL project also distributes the
OpenSSL FIPS module. This is a US government certified cryptography module that
works in conjunction with the main OpenSSL library. Most users \emph{should not} use
the FIPS module. If you \emph{must} use it then you should carefully read the notes
about FIPS on the OpenSSL project website,
\url{https://www.openssl.org/docs/fipsnotes.html}.

\section{Prerequisites for compiling OpenSSL from source}

In order to compile OpenSSL you will need to ensure that you have a suitable set of
tools that are required. All compilation typically takes place from the command for
your platform.

Firstly you will need a C compiler. Almost any modern C compiler should be suitable for
this. Common ones that are used include gcc and clang. On Linux most distributions will
provide a suitable package to install this through your package manager. XXXX give
details for RedHat/Debian??? XXXX

On the Windows platform there are two primary options for a C compiler. Either you can
use Microsoft's C compiler (as distributed with VisualStudio) or you can use the MinGW
gcc compiler. If you are going to use the compiler provided with VisualStudio then you
will still need to compile from the command line. VisualStudio provides the ability to
start a Developer Command Prompt, which sets up various environment variables to
correctly use the command line build tools. Start the version of the Developer
Command Prompt that matches the architecture for the platform you are taregetting. For
example if you want to build 32-bit OpenSSL then ensure that you start the 32-bit
Developer Command Prompt. Similarly if you want 64-bit OpenSSL then start the 64-bit
version.

If you plan to use the MinGW compiler then you should install MSYS2. This includes
packages for the the MinGW compiler in it \url{www.msys2.org}. Make sure you read the
MSYS2 introduction page at \url{https://github.com/msys2/msys2/wiki/MSYS2-introduction}
and follow the instructions for setting up MSYS2 with access to the correct compilers.
After installing MSYS2 with the correct compilers then, if you want to build 32-bit
OpenSSL, start the MinGW-w64 32-bit Shell or, to build 64-bit OpenSSL, start the
MinGW-w64 64-bit Shell.



\section{Compiling and Installing from source}

You can obtain the currently supported versions of OpenSSL from the download page at
\url{https://www.openssl.org/source}. Older versions are also available via a link
from that page.

Click on the link for the version you want to install in order to download an archive
file containing all of the relevant source code. The download archive is in ".tar.gz"
format. Many common archive/compression tools are able to read this format. Select a
suitable tool for your platform and uncompress and extract the files in the archive
into a suitable directory. For example on Linux you might type the following:

$ tar xvf openssl-1.1.0f.tar.gz

