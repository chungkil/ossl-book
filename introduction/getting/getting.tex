\chapter{Getting OpenSSL}

There are multiple ways to get hold of OpenSSL. This chapter will cover
how to compile and install OpenSSL from source, as well as discuss various
options for obtaining OpenSSL in pre-built binary form.

\section{Understanding OpenSSL version numbers}

The OpenSSL project (\url{https://www.openssl.org}) develops and
maintains the OpenSSL source code. The project stores all of the source
code in a source code control system known as git. Periodically the
project will generate a new \emph{release}. This is a snapshot of the source
code at a given point in time that is collected together into a single archive
file and made available for download.

At any one time there may be multiple versions of OpenSSL that are receiving
support by the project team. Support in this context means that bugs are
fixed and security issues are investigated, reported and fixed as appropriate.

For example OpenSSL 1.1.0 and OpenSSL 1.0.2 are two versions of OpenSSL. Each
version of OpenSSL may have multiple releases associated with it. These are
identified by a letter after the version number. The very first release for
a version has no letter. For example the first release for OpenSSL 1.1.0 has
the version number 1.1.0, the second one is 1.1.0a, the third one is 1.1.0b,
and so on. The letter releases contain bug and security fixes for the given
OpenSSL version (but never new features). To ensure that you always have the
latest fixes you should always try to use the latest letter release that is
available for the OpenSSL version that you are using. Letter releases for a
version are always fully backward compatible.

Versions of OpenSSL which have the same first two numbers are backwards source
and binary compatible with each other\footnote{This only applies if both versions
have been compiled with the same options}. For example 1.0.2 is backwards
source and binary compatible with 1.0.1 and 1.0.0. This means upgrading from
one of these versions to a later one should simply be a matter of upgrading the
library itself. No recompilation of application binaries using OpenSSL should be
necessary. If recompilation of applications is done then they should continue to
work as before. New features will typically have been added if the third number
has been increased.

Versions of OpenSSL which change the middle number are not backwards source or
binary compatible with each other. For example 1.1.0 is not fully backwards
source or binary compatible with 1.0.2. The changes required to upgrade an
application that works with 1.0.2 for 1.1.0 are relatively small and straight
forward though.

The OpenSSL project has not defined what it means if the first number changes,
but we should assume that this would signal some very significant change. The
project has no current plans for a version that changes this number.

OpenSSL versions are supported for a given period of time. Some OpenSSL versions
are designated as Long Term Support (LTS) versions. Non-LTS versions are supported
for 2 years. LTS versions are supported for 5 years. During the final year of
any support period the version only receives security fixes (but not bug fixes).
OpenSSL 1.0.2 is an LTS release.

Note that it is usually possible to have multiple versions of OpenSSL installed on
the same system at the same time.

New applications that are being developed to use OpenSSL should typically use the
latest version of OpenSSL available. This guide describes the use of OpenSSL 1.1.x
(where "x" is any number). It does not describe how to use OpenSSL 1.0.2 (or earlier)
which has a number of differences.

In addition to the main OpenSSL versions the OpenSSL project also distributes the
OpenSSL FIPS module. This is a US government certified cryptography module that
works in conjunction with the main OpenSSL library. Most users \emph{should not} use
the FIPS module. If you \emph{must} use it then you should carefully read the notes
about FIPS on the OpenSSL project website,
\url{https://www.openssl.org/docs/fipsnotes.html}.

\section{Using a Pre-Built OpenSSL Binary}

The OpenSSL Project does not distribute pre-built binaries, only source code.
Nevertheless there are a number of third parties that distribute pre-built
versions of OpenSSL. The project does not officially endorse any third party
distribution although it does maintain a page linking to some of them. The main
advantage of using a pre-built binary version is that it is typically much
easier and quicker to install in this way.

If you are using OpenSSL on Linux then it is almost always possible to obtain
OpenSSL from the package manager for your distribution.  If the binary comes
from a package manager, then usually the package maintainer will backport any
security fixes to the package and you will get the latest security fixes by
simply installing the latest updates. Often the package maintainer will apply
distribution specific patches to the code in order to make it work more
effectively with that distribution. This may mean that attempting to use a
source compiled version of OpenSSL in conjunction with system binaries designed
to work with the system specific pre-built version may not always work as
expected. Often the OpenSSL files may be split across multiple packages. For
example, on Debian systems, the command line OpenSSL tools are in the
``openssl'' package, the header files required for developing your own
applications are in the ``libssl-dev'' package, and the documentation is in the
``libssl-doc'' package.

The disadvantage of using a pre-built binary is that you have no control over
the build process itself. Therefore you will not be able to configure the build
specifically for your needs (for example in order to enable non-default options).
Additionally the pre-built binary version for your platform may not be the
latest available version with the latest features and/or the latest
(non-security) bug fixes. If you use a source version though you will need to
make sure you keep it regularly updated with any security fixes that might be
released from time to time.

It is usually possible to have more than one version of OpenSSL installed at the
same time. For example it is common to use the platform specific binary version
for built-in system binaries, but also have a source compiled version for use by
your own custom application.

There are a number of pre-built binaries available for Windows and other
platforms. The project maintains a list of these here:

\url{https://wiki.openssl.org/index.php/Binaries}

\section{Prerequisites for compiling OpenSSL from source} \label{sec:getting-prereq}

In order to compile OpenSSL you will need to ensure that you have a suitable set of
tools. All compilation typically takes place from the command line for your platform.

\subsection{C Compiler}

Firstly you will need a C compiler. Almost any modern C compiler should be suitable for
this. Common ones that are used include gcc and clang. On Linux most distributions will
provide a suitable package to install this through your package manager. On Debian
based systems you can use this:

\todo{Validate the various Debian/RedHat instructions on a fresh install}

\begin{verbatim}
$ sudo apt-get install build-essential
\end{verbatim}

On a RedHat based system you could use:

\begin{verbatim}
$ sudo yum install gcc
\end{verbatim}

On the Windows platform there are two primary options for a C compiler. Either you can
use Microsoft's C compiler (as distributed with VisualStudio) or you can use the MinGW
gcc compiler. If you are going to use the compiler provided with VisualStudio then you
will still need to compile from the command line. VisualStudio provides the ability to
start a Developer Command Prompt, which sets up various environment variables to
correctly use the command line build tools. Start the version of the Developer
Command Prompt that matches the architecture for the platform you are targetting. For
example if you want to build 32-bit OpenSSL then ensure that you start the 32-bit
Developer Command Prompt. Similarly if you want 64-bit OpenSSL then start the 64-bit
version.

If you plan to use the MinGW compiler then you should install MSYS2. This includes
packages for the the MinGW compiler in it \url{http://www.msys2.org}. Make sure you read the
MSYS2 introduction page at \url{https://github.com/msys2/msys2/wiki/MSYS2-introduction}
and follow the instructions for setting up MSYS2 with access to the correct compilers.
After installing MSYS2 with the correct compilers then, if you want to build 32-bit
OpenSSL, start the MinGW-w64 32-bit Shell or, to build 64-bit OpenSSL, start the
MinGW-w64 64-bit Shell.

\subsection{Perl} \label{sec:getting-prereq-perl}

Perl is used by multiple scripts within the OpenSSL build system. You will need perl
version 5.10.0 or later. On Linux you should install this through your package manager.
You can check the version of Perl that you have by running the following command:

\begin{verbatim}
$ perl -version
\end{verbatim}

Be careful to install the correct perl packages. OpenSSL uses a number of core perl
modules. Full details are in the file NOTES.PERL available in the top level directory
of the source code archive. On Debian based systems installing the "perl" package
should give you everything you need. On RedHat based systems install the "perl-core"
package.

On Windows you should use a version of perl appropriate to your build environment. If
you are using MSYS2 then you should use the MSYS2 perl package. From the MSYS2 shell
type:

\begin{verbatim}
$ pacman -S perl
\end{verbatim}

If you are using VisualStudio then select a suitable Windows Perl package such as
 ActiveState's Perl \url{https://www.activestate.com/activeperl} or Strawberry Perl
\url{http://strawberryperl.com/}. Download the appropriate file from the website and
run the installer package. Ensure that the Perl executable location is included in
the \%PATH\% environment variable.

Do not use the wrong perl version for a build environment. For example attempting to
use ActiveState Perl from MSYS2 or MSYS2 Perl from the VisualStudio developer prompt
will fail. All of these Windows Perl packages should come with the correct perl modules.

\subsection{Assembler, make and other build tools} \label{sec:getting-prereq-assem}

On Linux you will need a collection of other basic build tools (such as ``GNU as'',
``ld'', ``ar'' etc). These are distributed as part of the GNU binutils package
which can be installed as follows for Debian based systems:

\begin{verbatim}
$ sudo apt-get install binutils
\end{verbatim}

Or like this on RedHat based systems:

\begin{verbatim}
$ sudo yum install binutils
\end{verbatim}

On Linux you will also need ``make''. If you installed the ``build-essential''
package on Debian based systems to get your compiler then this will also install
make. On RedHat based systems you can install it as follows:

\begin{verbatim}
$ sudo yum install make
\end{verbatim}

On Windows everything you need should be included in your MSYS or VisualStudio
environment with the exception of an assembler. You will need to download and
install the NASM assembler for this (\url{http://www.nasm.us/}). Select the
appropriate Windows installer for dowload and then run it. Ensure that the
NASM executable location is included in your \%PATH\% environment variable. Note
that the Microsoft Assembler that comes with VisualStudio is not supported for
building OpenSSL.

\section{Compiling and Installing from source}

You can obtain the currently supported versions of OpenSSL from the download page at
\url{https://www.openssl.org/source}. Older versions are also available via a link
from that page.

Click on the link for the version you want to install in order to download an archive
file containing all of the relevant source code. The download archive is in ".tar.gz"
format. Many common archive/compression tools are able to read this format. Select a
suitable tool for your platform and uncompress and extract the files in the archive
into a suitable directory. For example on Linux you might type the following:

\begin{verbatim}
$ tar xvf openssl-1.1.0f.tar.gz
\end{verbatim}

Note that this guide only provides instructions for building OpenSSL 1.1.x (where
``x'' is any number). Other versions will need to be built differently.

\subsection{Configuring the OpenSSL build}

OpenSSL uses a perl script called \verb!Configure! to initialse the source directory
ready for building. This script is in the top-level directory of the source code
tree. Its primary function is to create the Makefile required for your specific
platform and build options.

In its simplest form it takes a single argument - an identifier (known as a
target) specific to your platform. For example to build for 64-bit Windows using
VisualStudio you would use the \verb!VC-WIN64A! target:

\begin{verbatim}
$ perl Configure VC-WIN64A
\end{verbatim}

Some common targets are listed in table \ref{tab:configure-targets}.

\begin{table}[tb]
\centering
\begin{tabular}{|l|l|}
\hline
\rowcolor{LightGray}
Target & Description \\
\hline
linux-x86\_64 & 64-bit Linux for the x86 architecture \\
\hline
linux-x86 & 32-bit Linux for the x86 architecture \\
\hline
mingw & 32-bit MSYS2 Windows \\
\hline
mingw64 & 64-bit MSYS2 Windows \\
\hline
VC-WIN32 & 32-bit VisualStudio Windows \\
\hline
VC-WIN64A & 64-bit VisualStudio Windows \\
\hline
\end{tabular}
\caption{Common targets for Configure}
\label{tab:configure-targets}
\end{table}

A full list of all available targets can be obtained by invoking \verb!Configure!
with no arguments:

\begin{verbatim}
$ perl Configure
\end{verbatim}

On many Linux/Unix platforms it is possible to auto-detect the correct target
platform. To do this you can use the provided shell script \verb!config!:

\begin{verbatim}
$ ./config
\end{verbatim}

This will attempt to work out the correct target platform and then invoke
\verb!Configure! accordingly. This may not work on all platforms but should be fine
on most popular ones.

The \verb!Configure! script can take a number of options which enable you to
customise the build. List the options you wish to use on the command line after
\verb!Configure!. For example to build with debugging symbols without any support
for deprecated APIs, you can use the following:

\begin{verbatim}
$ perl Configure --debug no-deprecated linux-x86_64
\end{verbatim}

Any options provided to the \verb!config! script will be passed down to the underlying
call to Configure so this also works:

\begin{verbatim}
$ ./config --debug no-deprecated
\end{verbatim}

A complete list of the supported options for Configure and a description of what
they do is available in the ``INSTALL'' file in the top level source directory.

\subsection{Building, Testing and Installing}

Once configured the build process is simply a matter of invoking the appropriate
\verb!make! command for your platform from the top level source directory. On most
platforms this is simply \verb!make!. For Windows VisualStudio this is called
\verb!nmake! (replace \verb!make! with \verb!nmake! in the examples below for that
platform).

\begin{verbatim}
$ make
\end{verbatim}

Assuming the build was successful then you can (optionally) invoke the OpenSSL
self-tests. This runs a large number of tests (which may take some time) to
verify that OpenSSL is operating as expected:

\begin{verbatim}
$ make test
\end{verbatim}

To install OpenSSL you use the following command:

\begin{verbatim}
$ make install
\end{verbatim}

This will copy all the OpenSSL files to the installation directory. Frequently
(on Linux/Unix) the user building OpenSSL does not have the relevant permissions
to install to the target directory, so you may need to perform this step as a
different user. For example:

\begin{verbatim}
$ sudo make install
\end{verbatim}

By default on Linux/Unix OpenSSL will install to \verb!/usr/local!. On
Windows the default is \verb!C:\Program Files\Common Files\SSL! or \\
\verb!C:\Program Files (x86)\Common Files\SSL!. This default can be changed
using the \verb!--prefix! option to \verb!Configure!.

\subsection{Building from Git}

Sometimes you may want to use an unreleased version of OpenSSL. For example
maybe you want to use the latest development version of the code in order for your
application to use some as yet unreleased feature. Or perhaps you have
encountered a bug that has been fixed by the OpenSSL development team but is not
yet in an official OpenSSL release.

In order to do this you will have to obtain the code for OpenSSL directly from
the source code management system (\verb!git!), rather than downloading it from
the downloads page. Warning: unreleased versions of OpenSSL are not suitable for
use in production environments. Development versions of the code are liable to
break from time to time. Importantly any security issues that are discovered in
development code which do not affect official releases (e.g. because they only
affect newly added or changed code) will just be fixed. These security issues
will not be advertised in any official project Security Advisories.

To obtain an unreleased version of OpenSSL you must first install \verb!git!.
This is available in most package managers for Linux, or directly from the
website for other platforms (\url{https://git-scm.com/}). Once installed you
can obtain the latest code through the following commands:

\begin{verbatim}
$ git clone git://git.openssl.org/openssl.git
$ cd openssl
\end{verbatim}

This will copy the latest OpenSSL code into the \verb!openssl! directory.
OpenSSL is primarily developed on the Linux platform. Therefore, all of the
source files use "Unix" style line endings. On the Windows platform you should
set git to use the correct line endings style:

\begin{verbatim}
$ git config core.autocrlf false
$ git config core.eol lf
$ git checkout .
\end{verbatim}

\verb!git! organises the source code into ``branches''. Each major version of
OpenSSL is in a separate branch. The default ``master'' branch contains the
newest, as yet unreleased, version of OpenSSL. To get the latest unreleased
fixes for OpenSSL 1.1.0 you should checkout that branch:

\begin{verbatim}
$ git checkout OpenSSL_1_1_0-stable
\end{verbatim}

Once the correct branch has been checked out, building OpenSSL from git is the
same as building from a downloaded archive file.

\section{Troubleshooting}

Sometimes things will go wrong during compilation and/or installation of
OpenSSL. Usually problems are caused by some environmental issue. In particular
carefully review the content of section \ref{sec:getting-prereq}. Some common
problems and their causes are listed below. If that still doesn't help for you
then you can send a question to the \verb!openssl-users! email list asking for
help (see \url{https://mta.openssl.org} for details). There are many very
experienced OpenSSL users on that list who may be able to help you.

\subsection{NASM not found}

Running \verb!Configure! results in the following error message:

\begin{verbatim}
NASM not found - please read INSTALL and NOTES.WIN for further details
\end{verbatim}

On Windows it is a pre-requisite that you have the NASM assembler installed (see
section \ref{sec:getting-prereq-assem}). If you have already installed NASM and
are still getting this error then, most likely, you have forgotten to update
your \%PATH\% environment variable to include the location of the NASM
executable. Update the \%PATH\% environment variable and then restart your
developer command prompt to pick up the change.

\subsection{Perl implementation doesn't support the right type of paths}

There are two variants of this error message. One where the Perl implementation
doesn't support Windows like paths:

\begin{verbatim}
This perl implementation doesn't produce Windows like paths (with backward
slash directory separators).  Please use an implementation that matches your
building platform.
\end{verbatim}

And another where the Perl implementation doesn't support Unix like paths:

\begin{verbatim}
This perl implementation doesn't produce Unix like paths (with forward slash
directory separators).  Please use an implementation that matches your
building platform.
\end{verbatim}

The former is caused by attempting to use MSYS2 perl (or some other similar perl
implementation) that expects paths to have forward slash directory separators to
perform a VisualStudio build. Change to a suitable perl implementation, and
ensure that perl implementation exists on your \%PATH\% environment variable
before any other version of perl you may have installed. See section
\ref{sec:getting-prereq-perl}.

The latter is caused by attempting to use some generic Windows perl
implementation in an MSYS2 build. For an MSYS2 build you must use the MSYS2
version of perl. See section \ref{sec:getting-prereq-perl}.

\subsection{Module machine type conflicts with target machine type}

During the build stage you encounter this error:

\begin{verbatim}
crypto\aes\aes_cfb.obj : fatal error LNK1112: module machine type 'x64' conflict
s with target machine type 'X86'
NMAKE : fatal error U1077: 'link' : return code '0x1'
Stop.
NMAKE : fatal error U1077: '"C:\Program Files (x86)\Microsoft Visual Studio 12.0
\VC\BIN\amd64\nmake.exe"' : return code '0x2'
Stop.
\end{verbatim}

This is caused by configuring OpenSSL for the \verb!VC-WIN32! target but
attempting to build it from a 64-bit Developer Command Prompt. Either change the
target to a 64-bit one (\verb!VC-WIN64A!) or use the 32-bit Developer Command
Prompt instead.

A similar version of this error can occur if OpenSSL is configured for a 64-bit
build but you are attempting to build it from a 32-bit Developer Command Prompt.

\begin{verbatim}
crypto\aes\aes_cfb.obj : fatal error LNK1112: module machine type 'X86' conflict
s with target machine type 'x64'
NMAKE : fatal error U1077: 'link' : return code '0x1'
Stop.
NMAKE : fatal error U1077: '"C:\Program Files (x86)\Microsoft Visual Studio 12.0
\VC\BIN\nmake.exe"' : return code '0x2'
Stop.
\end{verbatim}

